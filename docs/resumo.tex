\documentclass[a4paper,12pt]{article}

\usepackage{times}
\usepackage[utf8]{inputenc}
\usepackage[brazil]{babel}
\usepackage[a4paper,margin=2cm,columnsep=1cm]{geometry}
\usepackage{authblk}
\usepackage{titlesec}
\usepackage[pdftex]{graphicx}
\usepackage{mathtools}
\usepackage{amsmath}
\usepackage{enumitem}
\usepackage[font={footnotesize}]{caption}
\usepackage{bm}
\usepackage{sectsty}


\topmargin      0.0cm
\headheight     0.0cm
\headsep        0.0cm
\oddsidemargin  0.0cm
\evensidemargin 0.0cm
\textheight     22.86cm
\textwidth      16.51cm

\graphicspath{{images/}}
\allsectionsfont{\raggedright}
\titleformat*{\section}{\normalsize\bfseries\filcenter}
\titleformat*{\subsection}{\normalsize\bfseries\filcenter}

\renewcommand{\figurename}{\small Figure}
\newcommand{\figureref}[1]{Fig. (\ref{fig:#1})}
\newcommand{\equationref}[1]{Eq. (\ref{eq:#1})}
\newcommand{\bigsum}{\displaystyle\sum}
\newcommand{\twopartdef}[4]
{
    \left\{
        \begin{array}{ll}
            #1 & \mbox{se } #2 \\
            #3 & \mbox{se } #4
        \end{array}
    \right.
}
\newcommand{\tworowsmatrix}[2]
{
    \begin{bmatrix}
            #1\\
            #2
    \end{bmatrix}
}


\begin{document}

\title{\textbf{Resumo de Aprendizagem de Máquina 2014-2}}
\author{
    \textbf{Eduardo M. B. de A. Tenório}\\
    \small{\texttt{embat@cin.ufpe.br}}
}
\affil{\large CIn-UFPE}
\date{}

\maketitle


\begin{abstract}
\begin{itshape}
Este documento tem por finalidade ser um resumo dos assuntos abordados na disciplina \emph{Aprendizagem de Máquina} do período 2014-2 do CIn-UFPE, ministrada pelos professores Francisco Carvalho e Teresa Ludermir. A maioria do documento referencia o livro ``Pattern Classification", de Duda, Hart \& Stork. Os códigos utilizados como exercício de fixação encontram-se em \normalfont{github.com/embatbr/resumo-aprendizagem}.
\end{itshape}
\end{abstract}

\input{secao01}
\section{Estimação Paramétrica}

\subsection{Introdução}

Sabendo $P(\omega_i)$ e $p(\boldsymbol{x}|\omega_i)$ é possível projetar um classificador ótimo. Contudo, em aplicações reais de classificação de padrões raramente tem-se este conhecimento completo a respeito da estrutura probabilística do problema. Geralmente o que está disponível é um conhecimento superficial da situação e um conjunto de \textbf{dados de treinamento}.

Como exemplo, sejam 300 objetos divididos em 2 classes de modo que a classe 1 tenha 200 objetos e a classe 2 tenha 100. Cada objeto é gerado a partir de uma das 3 gaussianas bi-variadas:

\begin{description}\itemsep0pt
    \item[1a:] $\mu_1 = 60$, $\mu_2 = 30$, $\sigma_1^2 = 9$ e $\sigma_2^2 = 144$
    \item[1b:] $\mu_1 = 52$, $\mu_2 = 30$, $\sigma_1^2 = 9$ e $\sigma_2^2 = 9$
    \item[2:] $\mu_1 = 45$, $\mu_2 = 22$, $\sigma_1^2 = 100$ e $\sigma_2^2 = 9$
\end{description}

\noindent Nota-se que $\boldsymbol{\Sigma}_{1a}$, $\boldsymbol{\Sigma}_{1b}$ e $\boldsymbol{\Sigma}_2$ são matrizes diagonais, reduzindo-as a vetores de variância, o que significa independência entre as variáveis.

As duas imagems em \figureref{samples} mostram os dados de treinamento. Na de cima os dados são divididos em clusters gerados pelas gaussianas definidas anteriormente. Na segunda são dividos em duas classes.

\begin{figure}[ht]
    \centering
    \includegraphics[width=0.5\columnwidth]{samples}
    \caption{Classes 1a (verde), 1b (amarelo) e 2 (azul).}
    \label{fig:samples}
\end{figure}

Pode-se utilizar estas amostras para estimar as probabilidades e densidades de probabilidade desconhecidas. Para problemas de aprendizagem supervisionada, a estimação das $P(\omega_i)$ é bastante simples. Sabendo a frequência relativa de aparição de cada classe $\omega_i$ no conjunto de treinamento, pode-se estimar suas $P(\omega_i)$ (desde que o conjunto de treinamento seja uma boa representação do ``mundo real"). Contudo, estimar as $p(\boldsymbol{x}|\omega_i)$ é complicado quando se tem poucos dados, especialmente quando a dimensionalidade de $\boldsymbol{x}$ aumenta. Uma maneira mais prática é assumir cada $p(\boldsymbol{x}|\omega_i)$ como uma gaussiana, cujas média e matriz de covariância são dadas por $\boldsymbol{\mu}_i$ e $\boldsymbol{\Sigma}_i$. Logo, o problema passa de estimar uma função desconhecida $p(\boldsymbol{x}|\omega_i)$ para estimar os \textbf{parâmetros} $\boldsymbol{\mu}_i$ e $\boldsymbol{\Sigma}_i$.

Para a estimação dos parâmetros, será utilizada a técnica da \textbf{máxima-verossimilhança}.

\subsection{Máxima-Verossimilhança}

Estimação por Máxima-Verossimilhança (Maximum-Likelihood, ML) considera os parâmetros como quantidades cujos valores são fixos, mas desconhecidos. A melhor estimativa de seus valores é aquela que maximiza a probabilidade de obter as amostras observadas. Este método possui uma gama de atributos atrativos. Primeiramente, possui quase sempre boas propriedades de convergência quando o número de amostras de treinamento aumenta. Além disso, ML geralmente é mais simples que métodos alternativos, tais como técnicas bayesianas.

\subsubsection*{Princípio Geral}

Um conjunto $\mathcal{D}$ de amostras é dividido em conjuntos $\mathcal{D}_1, .., \mathcal{D}_c$ cujas amostras são desenhadas independentemente, de acordo com $p(\boldsymbol{x}|\omega_j)$, resultado de variáveis aleatórias independentes e identicamente distribuídas. Assume-se que $p(\boldsymbol{x}|\omega_j)$ tem uma forma paramétrica, como por exemplo $N(\boldsymbol{\mu}_j, \boldsymbol{\Sigma}_j)$, de modo que pode ser escrita como $p(\boldsymbol{x}|\omega_j, \boldsymbol{\theta}_j)$, onde $\boldsymbol{\theta}_j$ consiste nos componentes de $\boldsymbol{\mu}_j$ e $\boldsymbol{\Sigma}_j$. O problema torna-se utilizar as informações dadas pelas amostras para obter boas estimativas dos vetores desconhecidos $\boldsymbol{\theta}_j$, para cada $\omega_j$. Uma simplificação é assumir que as amostras de cada $\mathcal{D}_j$ provêem informações relevantes apenas aos seus respectivos $\boldsymbol{\theta}_j$.

\begin{figure}[ht]
    \centering
    \includegraphics[width=0.5\columnwidth]{maximum-likelihood}
    \caption{Estimação do parâmetro $\boldsymbol{\hat{\theta}}$, o valor $\boldsymbol{\hat{\theta}}$ achado e $l(\boldsymbol{\hat{\theta}})$.}
    \label{fig:maximum-likelihood}
\end{figure}

Usa-se o conjunto $\mathcal{D}$, composto pelas amostras de treinamento $\boldsymbol{x}_1, ..., \boldsymbol{x}_n$ desenhadas independentemente de $p(\boldsymbol{x}|\boldsymbol{\theta})$, para estimar os $\boldsymbol{\theta}$ desconhecidos, tal que

\begin{equation}
    p(\mathcal{D}|\boldsymbol{\theta}) = \prod_{k=1}^{n} p(\boldsymbol{x}_k|\boldsymbol{\theta}).
    \label{eq:pdf_D}
\end{equation}

\noindent A densidade $p(\mathcal{D}|\boldsymbol{\theta})$ é conhecida como a \textbf{verossimilhança} de $\boldsymbol{\theta}$ em relação às amostras, e sua estimativa máxima é, por definição, o valor $\boldsymbol{\hat{\theta}}$ que maximiza  $p(\mathcal{D}|\boldsymbol{\theta})$ (\figureref{maximum-likelihood}).

Devido a ser monotonicamente crescente e mais fácil de trabalhar que a verossimilhança, o uso da log-verossimilhança

\begin{equation}
    l(\boldsymbol{\theta}) \equiv \ln p(\mathcal{D}|\boldsymbol{\theta}) = \sum_{k=1}^{n} \ln p(\boldsymbol{x}_k|\boldsymbol{\theta})
    \label{eq:log_likelihood}
\end{equation}

\noindent é preferível. Logo, a solução para $\boldsymbol{\hat{\theta}}$ é

\begin{equation}
    \boldsymbol{\hat{\theta}} = \arg\max_{\boldsymbol{\theta}} l(\boldsymbol{\theta}),
    \label{eq:hat_theta}
\end{equation}

\noindent onde a dependencia do conjunto de dados $\mathcal{D}$ fica implícito. Se o número de parametros a serem estimados é $p$, então $\boldsymbol{\theta} = (\theta_1, ..., \theta_p)^t$ e $\boldsymbol{\nabla}_{\boldsymbol{\theta}} = (\frac{\partial}{\partial \theta_1}, ..., \frac{\partial}{\partial \theta_p})^t$, e, pela \equationref{log_likelihood}

\begin{equation}
    \boldsymbol{\nabla}_{\boldsymbol{\theta}} l = \sum_{k=1}^{n} \boldsymbol{\nabla}_{\boldsymbol{\theta}} \ln p(\boldsymbol{x}_k|\boldsymbol{\theta}).
    \label{eq:gradient_theta_log}
\end{equation}

Então, um conjunto de condições necessárias para a estimação da máxima verossimilhança para $\boldsymbol{\theta}$ pode ser obtido do conjunto de $p$ equações

\begin{equation}
    \boldsymbol{\nabla}_{\boldsymbol{\theta}} l = \boldsymbol{0}.
    \label{eq:gradient_theta_log_equal_zero}
\end{equation}

Uma solução $\boldsymbol{\hat{\theta}}$ para \equationref{gradient_theta_log_equal_zero} pode ser uma máximo global, um máximo/mínimo local, ou (raramente) um ponto de inflexão de $l(\boldsymbol{\theta})$. Se todas as soluções forem achadas, é garantido que representa o máximo verdadeiro, senão deve-se checar todas as soluções individualmente (ou calcular derivadas de segunda ordem) para identificar qual é o ótimo global.

\subsubsection*{Gaussiana com $\boldsymbol{\mu}$ desconhecido}

Neste caso o vetor $\boldsymbol{\theta} = \boldsymbol{\mu}$, levando a $l(\boldsymbol{\theta}) = l(\boldsymbol{\mu}) \equiv \ln p(\mathcal{D}|\boldsymbol{\mu})$. Logo pela \equationref{log_likelihood},

\begin{equation}
    l(\boldsymbol{\mu}) = \sum_{k=1}^{n} \ln p(\boldsymbol{x}_k|\boldsymbol{\mu}),
    \label{eq:log_mu}
\end{equation}

\noindent e pela \equationref{gradient_theta_log},

\begin{equation}
    \boldsymbol{\nabla}_{\boldsymbol{\mu}} l = \sum_{k=1}^{n} \boldsymbol{\nabla}_{\boldsymbol{\mu}} \ln p(\boldsymbol{x}_k|\boldsymbol{\mu}).
    \label{eq:nabla_mu_log}
\end{equation}

\noindent Como $\boldsymbol{\nabla}_{\boldsymbol{\mu}} \ln p(\boldsymbol{x}_k|\boldsymbol{\mu}) = \boldsymbol{\Sigma}^{-1} (\boldsymbol{x}_k - \boldsymbol{\mu})$ e $\boldsymbol{\nabla}_{\boldsymbol{\mu}} l = 0$ (\equationref{gradient_theta_log_equal_zero}), então

\begin{equation}
    \sum_{k=1}^{n} \boldsymbol{\Sigma}^{-1} (\boldsymbol{x}_k - \boldsymbol{\hat{\mu}}) = \boldsymbol{0}.
    \label{eq:nabla_mu_log_equals_zero}
\end{equation}

\noindent Multiplicando por $\boldsymbol{\Sigma}$ e rearranjando os termos, tem-se

\begin{equation}
    \boldsymbol{\hat{\mu}} = \frac{1}{n} \sum_{k=1}^{n} \boldsymbol{x}_k.
    \label{eq:mu_optimum_case_1}
\end{equation}

Fica evidente que para o caso em que apenas $\boldsymbol{\mu}$ é desconhecido, a estimativa para a máxima verossimilhança é apenas a média das amostras, às vezes escrita como $\boldsymbol{\hat{\mu}}_n$ para clarificar sua dependência do número de amostras.

\subsubsection*{Gaussiana com $\boldsymbol{\mu}$ e $\boldsymbol{\Sigma}$ desconhecidos}

Este é o caso mais típico, quando $\boldsymbol{\mu}$ e $\boldsymbol{\Sigma}$ são desconhecidos. Desta forma, este parâmetros constituem as componentes do vetor paramétrico $\boldsymbol{\theta}$, ou seja $\boldsymbol{\theta} = (\boldsymbol{\mu}, \boldsymbol{\Sigma})^t$. Considerando o caso univariado, $\theta_1 = \mu$ e $\theta_2 = \sigma^2$. Então

\begin{equation}
    \ln p(x_k|\boldsymbol{\theta}) = -\frac{1}{2} \ln 2\pi\theta_2 - \frac{1}{2\theta_2}(x_k - \theta_1)^2
    \label{eq:log_likelihood_mu_sigma_univar}
\end{equation}

\noindent e seu gradiente é

\begin{equation}
    \boldsymbol{\nabla}_{\boldsymbol{\theta}} l = \boldsymbol{\nabla}_{\boldsymbol{\theta}} \ln p(x_k|\boldsymbol{\theta}) = \tworowsmatrix{\frac{1}{\theta_2}(x_k - \theta_1)}{-\frac{1}{2\theta_2} + \frac{(x_k - \theta_1)^2}{2\theta_2^2}}
    \label{eq:grad_log_likelihood_mu_sigma_univar}
\end{equation}

\noindent e pela \equationref{nabla_mu_log_equals_zero} tem-se

\begin{equation}
    \sum_{k=1}^n \frac{1}{\hat{\theta_2}}(x_k - \hat{\theta_1}) = 0
    \label{eq:grad_log_likelihood_mu_sigma_univar_theta_1}
\end{equation}

\noindent e

\begin{equation}
    -\sum_{k=1}^n \frac{1}{2\hat{\theta_2}} + \sum_{k=1}^n \frac{(x_k - \hat{\theta_1})^2}{2\hat{\theta_2}^2} = 0,
    \label{eq:grad_log_likelihood_mu_sigma_univar_theta_2}
\end{equation}

\noindent Rearrumando tudo, tem-se

\begin{equation}
    \hat{\mu} = \frac{1}{n} \sum_{k=1}^n x_k
    \label{eq:mu_optimum_case_2}
\end{equation}

\noindent e

\begin{equation}
    \hat{\sigma}^2 = \frac{1}{n} \sum_{k=1}^n (x_k - \hat{\mu})^2.
    \label{eq:sigma_optimum_case_2}
\end{equation}

Dando um salto de fé (embora esta seja demonstrável, diferente da cristã) e trocando os $\hat{\mu}$ e $\hat{\sigma}$ por $\boldsymbol{\hat{\mu}}$ e $\boldsymbol{\hat{\Sigma}}$, respectivamente, tem-se

\begin{equation}
    \boldsymbol{\hat{\mu}} = \frac{1}{n} \sum_{k=1}^n \boldsymbol{x}_k
    \label{eq:bold_mu_optimum_case_2}
\end{equation}

\noindent e

\begin{equation}
    \boldsymbol{\hat{\Sigma}} = \frac{1}{n} \sum_{k=1}^n (\boldsymbol{x}_k - \boldsymbol{\hat{\mu}})(\boldsymbol{x}_k - \boldsymbol{\hat{\mu}})^t.
    \label{eq:bold_sigma_optimum_case_2}
\end{equation}
\section{Misturas}

\subsection{Introdução}

Esta seção trata sobre aprendizado \textbf{não-supervisionado}, no qual as amostras não são rotuladas com suas classes. Existem pelo menos cinco razões básicas para utilizá-lo: (1) coletar e rotular um grande conjunto de dados pode ser extremamente custoso; (2) o procedimento pode ser efetuado na ordem inversa, treinando com um grande número de amostras não rotuladas e então utilizar supervisão para rotular os grupos criados; (3) em muitas aplicações as características dos padrões podem mudar com o tempo; (4) métodos não-supervisionados podem ser utilizados para encontrar características que serão úteis para a categorização; (5) em estágio iniciais de uma investigação, pode ser interessante efetuar análise exploratória de dados para ganhar algum \emph{insight} sobre a natureza ou estrutura dos mesmos.

\subsection{Mistura de Densidades}

Assume-se que é sabido a estrutura completa de probabilidades para o problema com a exceção dos valores de alguns parâmetros. Mais especificamente, faz-se as seguintes suposições:

\begin{enumerate}\itemsep0pt
    \item As amostras são provinientes de um número $c$ de amostras.
    \item $P(\omega_j)$, para $j = 1, ..., c$, são conhecidas.
    \item As formas das $p(\boldsymbol{x}|\omega_j, \boldsymbol{\theta}_j)$, para $j = 1, ..., c$, são conhecidas.
    \item Os valores dos vetores $\boldsymbol{\theta}_1, ..., \boldsymbol{\theta}_c$ são desconhecidos.
    \item Os rótulos são desconhecidos.
\end{enumerate}

Assume-se que as amostras são obtidas selecionando primeiramente a classe $\omega_j$ com probabilidade $P(\omega_j)$ e então selecionando um $\boldsymbol{x}$ de acordo com a lei de probabilidade $p(\boldsymbol{x}|\omega_j, \boldsymbol{\theta}_j)$. Então, a função de densidade de probabilidade para as amostras é dada por

\begin{equation}
    p(\boldsymbol{x}|\boldsymbol{\theta}) = \sum_{j=1}^c p(\boldsymbol{x}|\omega_j, \boldsymbol{\theta}_j) P(\omega_j),
    \label{eq:pdf_mixtures}
\end{equation}

\noindent onde $\boldsymbol{\theta} = (\boldsymbol{\theta}_1, ..., \boldsymbol{\theta}_c)^t$. A função $p(\boldsymbol{x}|\boldsymbol{\theta})$ é chamada \textbf{densidade de mistura}. As densidades condicionais $p(\boldsymbol{x}|\omega_j, \boldsymbol{\theta}_j)$ são chamadas de \textbf{densidades componentes} e as $P(\omega_j)$ são chamadas de \textbf{parâmetros da mistura} (que podem ser incluídas entre os parâmetros desconhecidos).

O objetivo básico é usar as amostras desenhadas a partir da densidade de misturas para estimar o vetor paramétrico desconhecido $\boldsymbol{\theta}$. Sabendo $\boldsymbol{\theta}$ pode-se decompor a mistura em componentes e usar um classificador por \textbf{estimação a posteriori máxima} (maximum a posteriori, MAP) nas densidades derivadas. Para existir a possibilidade de solução, $\boldsymbol{\theta}$ só pode ter um valor.

\subsection{Estimação por Máxima Verossimilhança}

Supondo um conjunto $\mathcal{D} = \{\boldsymbol{x}_1, .., \boldsymbol{x}_n\}$ de amostras não rotuladas desenhadas independentemente a partir da densidade da \equationref{pdf_mixtures}. A verossimilhança das amostras observadas é dada pela \equationref{pdf_D}. Assumindo $p(\mathcal{D}|\boldsymbol{\theta})$ diferencial, seja $l$ o logaritmo da verossimilhança dado pela \equationref{log_likelihood}

\begin{equation}
    l(\boldsymbol{\theta}) = \sum_{k=1}^n \ln p(\boldsymbol{x}|\boldsymbol{\theta}) = \sum_{k=1}^n \ln \Bigg{(}\sum_{j=1}^c p(\boldsymbol{x}_k|\omega_j,\boldsymbol{\theta})P(\omega_j)\Bigg{)}.
    \label{eq:log_mixtures}
\end{equation}

\noindent e

\begin{equation}
    \boldsymbol{\nabla_{\theta_i}} l(\boldsymbol{\theta}) = \sum_{k=1}^n \frac{1}{p(\boldsymbol{x}_k|\boldsymbol{\theta})} \boldsymbol{{\nabla_{\theta_i}}} \Bigg{[}\sum_{j=1}^c p(\boldsymbol{x}_k|\omega_j,\boldsymbol{\theta})P(\omega_j)\Bigg{]}.
    \label{eq:gradient_theta_log_mixtures_1}
\end{equation}

\noindent Assumindo $\boldsymbol{\theta}_i$ e $\boldsymbol{\theta}_j$ independentes para $i \neq j$, e sendo a probabilidade a posteriori

\begin{equation}
    P(\omega_i|\boldsymbol{x}_k, \boldsymbol{\theta}) = \frac{p(\boldsymbol{x}_k|\omega_i, \boldsymbol{\theta}_i)P(\omega_i)}{p(\boldsymbol{x}_k|\boldsymbol{\theta})},
    \label{eq:posteriori_prob_theta}
\end{equation}

\noindent a \equationref{gradient_theta_log_mixtures_1} pode ser reescrita como

\begin{equation}
    \boldsymbol{\nabla_{\theta_i}} l = \sum_{k=1}^n P(\omega_i|\boldsymbol{x}_k, \boldsymbol{\theta}) \boldsymbol{\nabla_{\theta_i}} \ln p(\boldsymbol{x}_k|\omega_i, \boldsymbol{\theta}_i).
    \label{eq:gradient_theta_log_mixtures_2}
\end{equation}

\noindent Como $\boldsymbol{\nabla_{\theta_i}} l = \boldsymbol{0}$ para o valor de $\boldsymbol{\theta}_i$ que maximiza $l$, a estimativa da máxima-verossimilhança $\boldsymbol{\hat{\theta}}_i$ de satisfazer as condições

\begin{equation}
    \sum_{k=1}^n P(\omega_i|\boldsymbol{x}_k, \boldsymbol{\hat{\theta}}) \boldsymbol{\nabla_{\hat{\theta}_i}} \ln p(\boldsymbol{x}_k|\omega_i, \boldsymbol{\hat{\theta}}_i) = 0 \text{ para } i = 1, ..., c.
    \label{eq:gradient_hat_theta_log_mixtures_equals_zero}
\end{equation}

\noindent Dentre as soluções para estas equações para $\boldsymbol{\hat{\theta}}_i$ encontra-se aquela que maximiza a verossimilhança.

\end{document}